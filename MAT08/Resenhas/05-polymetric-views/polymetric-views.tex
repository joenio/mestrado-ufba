\documentclass[12pt]{article}
\usepackage[utf8]{inputenc}
\usepackage[brazil]{babel}
\usepackage{fancyvrb}
\usepackage[margin=0.7in]{geometry}
\bibliographystyle{ieeetr}


\title{Resenha: Polymetric Views - A Lightweight Visual Approach to Reverse Engineering\cite{PolymetricViews} \\
 \large MAT08 - Evolução de Software (2012.2)}
\author{Joenio Marques da Costa}
\date{06 de dezembro de 2012}

\begin{document}

\maketitle

\section{Resumo}

A engenharia reversa em grandes sistemas de software tem se tornado um assunto
de grande relevancia devido a sua importancia no auxílio a tarefas de
manutenção, re-engenharia e evolução. Manter e evoluir sistemas legados é uma
tarefa inerentemente difícil por diversas razões, estes sistemas são grandes,
maduros e complexos, e são resultado de um longo tempo de investimento e
precisam ser pagos. Além de sofrer problemas típicos, como a saída de
desenvolvedores, o uso de métodos de desenvolvimento ou linguagens
ultrapassadas, ou ainda documentação desatualizada, incompleta ou inexistente.
Diante disso fica claro que ter uma abordagem de apoio as tarefas de
manutenção, re-engenharia e evolução torna-se extremamente necessária afim de
reduzir custos e possibilitar a sua execução, com isto em mente é proposto
modelos de visualizações de software chamados de "polymetric views", uma
visualização de software simples enriquecida com métricas, esta combinação
oferece o poder de entendimento visual fornecido pelas visualizações aliado a
rica informação fornecida pelas métricas que usualmente são de difícil
interpretação quando sozinhas. Através de "polymetric views" foi criada uma metodologia de
engenharia reversa em sistemas de software que oferece como resultado final uma
visão geral de um sistema e joga luz em que partes estão boas ou ruins,
deixando a cargo do projetista fazer reengenharia nestas partes ou não. Este
resultado é alcançado através da ferramenta desenvolvida durante a pesquisa,
chamada CodeCrawler, uma ferramenta par avisualização de softwares... os casos
de uso e contribuicoes para os desenvolvedores originais. O resultado tipico de
todo estudo de caso é um relatório contendo conjunto de polymetri views e uma
lista de possiveis problemas e erros. Os desenvolvedores originais gostaram
destes relatorio e usualmente se utilizaram dos resultados para documentar seus
projetos.

\section{Questoes}

\begin{itemize}
  \item Como está o estado atual desta ferramente CodeCrowler? Ela pode ser utilzada com segurança?
  \item Esta proposta de visualização chamada "polymetric views" continua a ser utilizada em outros estudos?
\end{itemize}

\bibliography{bibliografia}
\end{document}
