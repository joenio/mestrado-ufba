\documentclass[12pt]{article}
\usepackage[utf8]{inputenc}
\usepackage[brazil]{babel}
\usepackage{fancyvrb}
\usepackage[margin=0.7in]{geometry}
\bibliographystyle{ieeetr}

\title{How Effective Developers Investigate Source Code: An Exploratory Study
 \cite{HowEffectiveDevelopers} \\
 \large MATE08 - Tópicos em Engenharia de Software 1 (2012.2)}
\author{Joenio Marques da Costa}
\date{24 de janeiro de 2013}

\begin{document}

\maketitle

\section*{Resenha}

Sistemas de software mudam, tais mudanças normalmente envolvem 3 fases:
entender, alterar e avaliar. Durante a fase de alteração os desenvolvedores
geralmente deparam-se com a necessidade de modificar partes distintas do
software, isto conduz a necessidade de investigar com antecedencia onde as
modificações serão feitas. Algumas ferramentas de apoio ao desenvolvimento
fornecem recursos de auxílio a esta atividade mas estão longe de garantir a
sua eficácia pois dependem fortemente da maneira como cada desenvolvedor as
utiliza. Identificar e avaliar os fatores que levam a uma investigação eficaz
sem levar em conta fatores comportamentais e do ambiente é um grande desafio
da engenharia de software. Na tentativa de vencer este desafio foi elaborado
um estudo com um grupo de desenvolvedores sob observação durante a modificação
de um projeto de software. A observação destes desenvolvedores juntamente com
os resultados produzidos permitiu descrever a seguinte teoria: {\it durante a
tarefa de investigação, um comportamento metódico é mais eficaz que uma
abordagem oportunista}. Esta teoria foi elaborada a partir da classificação
dos desenvolvedores em 2 grupos distintos, em um aqueles que obtiveram sucesso
na tarefa modificação e em outro aqueles que não obtiveram sucesso. Cada grupo
foi analisado e uma série de observações sobre fatores comportamentais que
levam ao sucesso ou ao insucesso foram elaboradas. Elaborar tais observações
esbarra em inúmeras barreiras comuns a trabalhos desta natureza, como por
exemplo a dificuldade em garantir a validade de estudos envolvendo múltiplos
participantes ou a possibilidade de introduz variáveis que não representam o
fenômeno observado. Ainda assim foi possível realizar o estudo com um certo
grau de validade, foi utilizado um software real de tamanho médio e uma tarefa
de modificação bastante realista. O estudo deixa como contribuição uma série
de observações sobre quais características contribuem para uma investigação de
software eficiente e elabora uma metodologia detalhada de como fazer estudos
empíricos com desenvolvedores visando fatores comportamentais. Em minha
opinião o estudo elabora uma teoria correta a respeito dos fatores
comportamentais que conduzem ao sucesso na tarefa de desenvolvimento de
software, estes fatores são realmente humanos, como conhecimento, habilidade e
experiência. A escolha do software e a tarefa de modificação se mostram
suficientes mas um grupo de apenas de apenas 5 desenvolvedores é pouco para
traçar afirmações gerais.

%A avaliação da qualidade da solução dada para cada desenvolvedor é bastante
%subjetiva uma vez que é feito por um 
%sido feita por um processo subjetivo, é extremamente válido ao meu ver avaliar
%deste ponto de vista. A engenharia não possui ainda uma fórmula para calcular
%a qualidade de um artefato, a única alternativa é realmente esta. Mas um grupo
%de apenas 5 pessoas pode ser pouco, visto que 

\bibliography{bibliografia}
\end{document}
