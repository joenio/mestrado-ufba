\documentclass[12pt]{article}
\usepackage[utf8]{inputenc}
\usepackage[brazil]{babel}
\usepackage{fancyvrb}
\usepackage[margin=0.7in]{geometry}
\bibliographystyle{ieeetr}

\title{Resenha: Concern Graphs: Finding and Describing Concerns Using
 Structural Program Dependencies\cite{ConcernGraphs} \\
 \large MAT08 - Evolução de Software (2012.2)}
\author{Joenio Marques da Costa}
\date{20 de dezembro de 2012}

\begin{document}

\maketitle

\section*{Resumo}

Um bom design num sistema de software envolve, entre outras coisas, em ter uma
boa modularização, isto permite principalmente que os desenvolvedores alterem o
sistema de maneira segura sem interferir em outras partes do mesmo.  Não é raro
no entando encontrarmos sistemas com designs inadequados e de difícil
manutenção e evolução, usualmente estas dificuldades são enfrentadas através de
ferramentas IDE, analisadores léxicos ou histórico de sistemas de controle de
versão. Estas abordagens ajudam a encontrar conceitos espalhados pelo
código-fonte, mas levam sempre ao mesmo resultado: o desenvolvedor é
apresentado a linhas de código-fonte que serão analisadas manualmente, este
processo é trabalhoso e dificulta a localização de conceitos. Com isto em mente
é proposto o {\it Grafo de Conceitos}, uma representação visual dos conceitos
encontrados no código-fonte, efetiva, de fácil criação, manipulação e análise.
Esta proposta foi validada através da criação de uma ferramenta chamada {\it
FEAT}, ela cria, analiza e visualiza os {\it Grafos de Conceitos} de forma
fácil e interativa. Em um cenário comum o desenvolvedor percorre os arquivos de
código-fonte em busca dos conceitos relacionados a manutenção. Através da
ferramenta {\it FEAT} o desenvolvedor apenas monta um modelo que será então
convertido num {\it Grafo de Conceito}. Para avaliar a efetividade da
ferramenta {\it FEAT} e os {\it Grafos de Conceitos} foram feitos vários
estudos de caso que comprovaram a eficácia ao dar ao desenvolvedor a visão
necessária ao fazer manutenção. Não vejo pontos fracos no artigo, mas não
concordo com a real necessidade de ferramentas de apoio neste sentido,
conhecimento técnico e experiência de desenvolvimento são os ingredientes
comuns para manter e evoluir sistemas de software.

\bibliography{bibliografia}
\end{document}
