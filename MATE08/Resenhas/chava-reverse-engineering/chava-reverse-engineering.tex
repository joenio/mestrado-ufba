\documentclass[12pt]{article}
\usepackage[utf8]{inputenc}
\usepackage[brazil]{babel}
\usepackage{fancyvrb}
\usepackage[margin=0.7in]{geometry}
\bibliographystyle{ieeetr}

\title{Chava: Reverse Engineering and Tracking of Java Applets
 \cite{Chava} \\
 \large MAT08 - Tópicos em Engenharia de Software 1 (2012.2)}
\author{Joenio Marques da Costa}
\date{17 de janeiro de 2013}

\begin{document}

\maketitle

\section*{Resenha}

No início a Internet era baseada em servidores Web provendo conteúdo HTML
estático, não demorou muito para surgirem o scripts CGI
e logo em seguida os applets Java oferecendo
interfaces ricas e processamento no lado do cliente. Existem muitas
ferramentas para análise estrutural de arquivos HTML e muitas outras para
análise estrutural entre os componentes de grandes programas. Mas a maioria
destas ferramentas não suportam applets Java, este trabalho apresenta então o
Chava, uma ferramenta para analisar applets Java, com suporte a verificação de
mudanças entre versões, análise de HTML conjuntamente com applets Java, além
de outros recursos. Chava adota um modelo de dados onde cada programa é visto
como um conjunto de entidades, que referenciam-se umas as outras. Este modelo armazena classes,
interfaces, pacotes, arquivos, métodos, campos e textos. Cada uma destas
entidades pode ter alguns atributos como parentesco e escopo e trazem relações
como herança e dependência. O uso da ferramenta se dá a partir da análise de
um programa seguido da criação de um banco de dados baseado no modelo de
dados, a partir daí é possível fazer consultas e visualizações a respeito do
programa, estas consultas e visualizações são realizadas através do CIAO, um
conjunto de ferramentas para consulta e visualização de software. Com ele é
possível agrupar entidades, visualizar diferença entre versões, exibir
graficamente o relacionamentos entre entidades, além de calcular métricas a
respeito da complexidade do programa. Chava é implementado em C e possui cerca
de 3 mil linhas de código, para medir a sua performance foi feito um estudo
comparativo entre ele, o javac e o javap utilizando como entrada os seguintes
projetos: a biblioteca padrão do Java, a biblioteca Swing e o projeto
WebDelta. Nos testes realizados o Chava se mostrou mais
eficiente que o javac e o javap. Com o grande aumento de soluções Java
torna-se importante ferramentas de apoio a atividades comuns da engenharia de
software, com o Chava é possível trabalhar com grandes programas e executar
analises complexas como análise de impacto e clustering.

Comparar versões distintas e exibir graficamente suas diferenças é um recurso
muito interessante mas a ferramenta como um todo me parece de pouca utilidade
prática uma vez que novas tecnologias Web seguindo padrões abertos vem
constantemente substituindo soluções como applets Java. Vejo recursos
interessantes no Chava que servem mais de inspiração para outras ferramentas,
o uso da ferramenta em sí pode ser útil na replicação ou validação do artigo
apresentado.

\bibliography{bibliografia}
\end{document}
