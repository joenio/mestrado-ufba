\documentclass[conference]{IEEEtran}
\usepackage[utf8]{inputenc}
\usepackage[brazil]{babel}
\usepackage{fancyvrb}
\bibliographystyle{ieeetr}

\title{
  Ferramentas desenvolvidas durante pesquisas em engenharia de software no Brasil:\\
  Uma análise dos últimos 15 anos de SBES
}

\author{
  \IEEEauthorblockN{Joenio Marques da Costa}
  \IEEEauthorblockA{Universidade Federal da Bahia (UFBA)\\
    Email: joenio@colivre.coop.br
  }
  \and
  \IEEEauthorblockN{Autor 2...}
  \IEEEauthorblockA{Twentieth Century Fox\\
    Email: homer@thesimpsons.com
  }
}

\begin{document}

\maketitle

\begin{abstract}
\dots
\end{abstract}

\section{Introdução}

%Muita ênfase é dada ao processo científico, mas em nossa área de ES é comum
%desenvolver-se ferramentas durantes as pesquisas, é notório que ferramentas
%são necessárias já que muitos pesquisadores as desenvolvem, mas porque é dado
%tão pouco enfoque as ferramentas? Elas não deveriam ser melhor avaliadas e
%também ter maior atenção por parte de quem desenvolve ou de quem avalia um
%certo trabalho de pesquisa?



% Cavalcanti et al\cite{Cavalcanti11} desenvolveram revisão sistemática das
% últimas 24 edições do SBES, utilizaram o método proposto por Kitchenham e
% Charters\cite{Kitchenham2007}...  O meu estudo seguirá o mesmo formato
% desenvolvido no trabalho de Calvacanti\cite{Cavalcanti11}, mas com a mudança
% de foco de pesquisas da área de engenharia de software no geral, para
% pesquisas em engenharia de software que propõe e/ou desenvolvem ferramentas de
% software. Objetivo de identificar informações quantitativas e qualitativas
% sobre a produção de ferramentas durantes pesquisas em engenharia de software,
% este objetivo é semelhante ao objetivo do trabalho de \cite{Cavalcanti11} mas
% com foco em ferramentas. Este trabalho de cavalcanti mostra também que há uma
% grande concentração de pesquisas apresentando ferramentas, métodos e processos
% submetidos ao SBES entre as edições avaliadas, 1987 à 2010.

% TRABALHO FUTURO
% Em \cite{Kon11} é citado que foram analisados 6773 projetos usando ferramentas
% automatizadas para avaliar a atratividade de um software a partir de métricas
% extraídas do código-fonte, este estudo:
% 
% A study of the relationships between source code metrics and attractiveness in
% free software projects.
% 
% Pode talvez ser replicado em projetos desenvolvidos pela comunidade acadêmica,
% isso vai avaliar se os projetos sao atrativos ou não.

% \cite{Kon11} cita também que ocorre o fato de algumas ferramentas serem
% disponibilizadas publicamente, mas o link divulgado não funciona mais, ou está
% fora do ar, ou não cita a ferramenta.
% 
% Kon11 também cita em sua conclusão que percebeu em seu estudo uma tendência no
% aumento de pesquisadores Brasileiros em liberar suas ferramentas como FLOSS,
% tendo como benefício para estas pesquisas facilidade de reprodução por outros
% pesquisadores, bem como oportunidade de receberem contribuições externas
% melhorando suas ferramentas.

% PROBLEMA QUE TALVEZ EXISTA ENTRE PESQUISAS QUE DESENVOLVEM FERRAMENTAS
% \cite{Leite11} conclui em uma revisão sistemática de literatura de 5 anos
% analizando as publicações do SBES entre os anos de 2006 e 2010, com objetivo
% de dar uma visão geral das publicações recentes do SBES e analisar a
% relevancia para a comunidade e industria, percebeu que há uma incipiente
% colaboração entre universidades nas pesquisas.

Pesquisadores de engenharia de software encontram dificuldades em utilizar
ferramentas desenvolvidas pela comunidade acadêmica em suas pesquisas,
estas dificuldades geram como consequência ao menos dois problemas:

\begin{itemize}
\item Duplicação de esforço
  %- {\it pesquisador não utiliza ferramenta existente e desenvolve sua própria solução}
\item Dificuldade em replicar estudos
  %- {\it pesquisador não consegue reproduzir pesquisas com exatidão}
%        algo necessário para aumentar a validade externa dos estudos
%        Kon \cite{Kon11} sugere que distribuir
%        ferramentas como software livre é requisito para evitar este problema
% TODO devo pesquisar sobre estudos que apontam a importancia de
        % reprodubilidade de pesquisas científicas de modo geral para dar
        % lastro a este argumento aqui
\end{itemize}

Para possibilitar resolver estes problemas é necessário entender quais fatores
geram dificuldades aos pesquisadores na utilização de ferramentas.

Com isto em mente será feita uma caracterização das ferramentas
desenvolvidas nos últimos 15 anos de pesquisas em engenharia de software no
Brasil, esta caracterização ocorrerá através de uma revisão sistemática de
literatura de trabalhos submetidos ao Simpósio Brasileiro de Engenharia de
Software (SBES), tal revisão será guiada a partir das seguintes questões sobre
cada ferramenta:

\begin{itemize}
\item Quando foi lançada publicamente?
\item Quando recebeu sua última atualização?
\item Qual pesquisador desenvolveu?
\item Qual grupo de pesquisa desenvolveu?
\item Dá suporte para quais sistemas operacionais?
\item Em qual linguagem de programação foi escrita?
\item Em quais momentos foi utilizada após o lançamento?
\item Qual pesquisador utilizou após o lançamento?
\item Quais grupos de pesquisa utilizaram após o lançamento?
\item Qual problema resolve, qual o domínio?
\item É utilizada fora do seu grupo de pesquisa?
\item É utilizada fora da comunidade acadêmica? É adotada pela indústria?
\item É derivada ou é extensão de alguma outra ferramenta?
\item Tem cobertura de testes?
\item Qual o nível da cobertura de testes?
\item Tem documentação?
\item Em quais idiomas a documentação está disponível?
\item Como se deu a evolução histórica das ferramentas em quantidade, autores
  e organizações no SBES? (questão adaptada do trabalho de
  \cite{Cavalcanti11})
\item Quais os autores mais ativos em publicação de ferramentas no SBES?
  (questão adaptada do trabalho de \cite{Cavalcanti11})
\item Como se deu a evolução de quantidade de ferramentas publicadas no SBES?
  (questão adaptada do trabalho de \cite{Cavalcanti11})
\item Quantos artigos fazem referência à ferramenta? Ou ao artigo que propôs a
  ferramenta?
\end{itemize}

Esta caracterização dará uma visão geral sobre as ferramentas desenvolvidas em
pesquisas de engenharia de software no Brasil, bem como permitirá entender
aspectos do ciclo de vida de cada ferramenta.

\section{Metodologia}

\begin{itemize}
\item Levantar revisões sistemáticas de literatura de engenharia de software\\
        {\it pode haver algum estudo anterior com a mesma abordagem e que
        responda ao mesmo problema}
\item Levantar revisões sistemáticas de literatura sobre ferramentas de
        engenharia de software\\
        {\it pode existir estudos similares ou complementares}
\item Iniciar revisão sistemática de literatura para responder ao problema
        definido\\
        {\it buscar estudos dos últimos 15 anos de pesquisas em engenharia de
        software}
\end{itemize}

\section{Conclusão}

\ldots

%Kon \cite{Kon11} conclui que construir ferramentas como software livre gera
%transferencia de tecnologia em várias vias, uma delas é que proporciona
%pesquisadores terem acesso a uma gama enorme de ferramentas existentes já
%construídas pela comunidade acadêmica, deixando assim espaço para que o
%pesquisador foque seu estudo em sua área de pesquisa, e não se desvie tendo
%que desenvolver suas próprias ferramentas, perdendo tempo que poderia ser
%usado para se aprofundar mais em suas pesquisas. (esta afirmação de Kon é algo
%que eu quero também concluir em minha pesquisa).

\bibliography{bibliografia}
\end{document}
