\documentclass[12pt]{article}
\usepackage[utf8]{inputenc}
\usepackage[brazil]{babel}
\usepackage{fancyvrb}
\usepackage[margin=0.7in]{geometry}
\bibliographystyle{ieeetr}


\title{CodeTimeline: Storytelling with Versioning Data\cite{CodeTimeline} \\
 \large MAT08 - Tópicos em Engenharia de Software 1 (2012.2)}
\author{Joenio Marques da Costa}
\date{06 de dezembro de 2012}

\begin{document}

\maketitle

\section*{Resenha}

Entender e trabalhar com sistemas de software geralmente requer conhecimento
sobre seu histórico e sobre as decisões arquiteturais tomadas no passado, este
conhecimento raramente é documentado e encontra-se usualmente espalhado na
memória dos desenvolvedores. Experiências na indústria mostram que requisitar
engenheiros de software para manter esta documentação tem falhado. Novas
abordagens utilizando aspectos de redes sociais tem surgido e se mostrado
bastante promissoras neste sentido, este trabalho propõe uma abordagem de
documentação baseada nestes aspectos utilizando {\it story-telling
visualization} com foco em grupos fechados de desenvolvedores. Surge assim o
{\sc CodeTimeline}, um protótipo de um ambiente de visualização e interação com
o histórico de projetos de software, nele é possível adicionar eventos em
pontos específicos da vida do projeto em forma de emails, conversas de
bate-papo, imagens, fotos, entre outros, sendo possível consultar a relação
entre tais eventos e as mudanças efetuadas no código-fonte, este protótipo
oferece 2 dimensões de visualização distintas: (a) {\it collaboration view} e
(b) {\it sourcecloud flow view}, das quais destaca-se a (b) por ser a principal
contribuição deste trabalho. {\it Sourcecloud flow view} é uma forma de
visualização construída com base nas palavras extraídas do código-fonte em
forma de uma nuvem, nesta nuvem múltiplas dimensões podem ser representadas
através do tamanho e cor da fonte, podendo representar por exemplo a tendência
de surgimento ou desaparecimento de uma certa palavra durante a vida do
projeto. Este protótipo funciona como um ponto central de interação entre o
desenvolvedor e o código-fonte do projeto, promovendo a possibilidade de
navegar entre os eventos com informações que antes estavam apenas na memória
dos desenvolvedores. Estudo interessante, com uma abordagem mais experimental
que de costume, a proposta é interessante mas me parece pouco usual por se
basear em grupos fechados de desenvolvedores, seria muito interessante ver uma
proposta desta natureza aplicada a projetos abertos, com grupos de
desenvolvedores grandes e descentralizados, onde os dados fossem publicamente
visíveis através de uma plataforma Web e a inclusão fosse possível a partir de
várias plataformas e ferramentas distintas.

\bibliography{bibliografia}
\end{document}
