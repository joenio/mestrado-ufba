\documentclass[12pt]{article}
\usepackage[utf8]{inputenc}
\usepackage[brazil]{babel}
\usepackage{fancyvrb}
\usepackage[margin=0.7in]{geometry}
\bibliographystyle{ieeetr}

\title{Resenha: Data Mining for Software Engineering \cite{DataMiningfo} \\
 \large MAT08 - Evolução de Software (2012.2)}
\author{Joenio Marques da Costa}
\date{29 de novembro de 2012}

\begin{document}

\maketitle

\section{Resumo}


A melhoria na produtividade e qualidade de software é uma importante meta da
Engenharia de Software, e a Mineração de Dados de Engenharia de Software tem
sido utilizada cada vez mais para atingir esta meta, a grande quantidade de
dados sobre projetos de software e seus processos de desenvolvimento tornam a
Mineração de Dados de Engenharia de Software um grande aliado na solução de
tarefas comuns da Engenharia de Software, como: localizar bugs, efetuar
manutenção ou desenvolver novos requisitos. No entando, a Mineração de Dados de
Engenharia de Software traz consigo uma série de novos desafios, como:
necessidade de adaptar ferramentas de mineração existentes, suprir demanda por
algoritmos de mineração capazes de trabalhar com o grande volume de dados
disponíveis, ou mesmo oferecer um meio de mineração sob demanda onde o tempo de
resposta seja imediato. Estes desafios podem ser superados através da
aproximação entre a comunidade de Mineração de Dados e a comunidade de
Engenharia de Software, esta pesquisa é um exemplo de uma iniciativa neste
sentido, tendo como resultado o desenvolvimento de uma série de novos
algoritmos de Mineracao de Dados capazes de lidar com dados da Engenharia de
Software.

É feito um relato muito esclarecedor sobre a Mineração de Dados em apoio a
tarefas de Engenharia de Software mostrando que mineração é um aliado
fundamental no estudo sobre Evolução de Software, uma vez que extração de
informações a partir de dados brutos é uma tarefa comum deste processo.

\section{Questoes}

\begin{itemize}
  \item É possível estudar Evolução de Software sem utilizar Mineração de
     Dados?
  \item A mineracao em dados texto me pareceu bem mais desafiadora que a
     mineração em dados de sequência ou grafos. Isto é verdade? Qual o estado
     dela em relação às demais?
\end{itemize}

\bibliography{bibliografia}
\end{document}
