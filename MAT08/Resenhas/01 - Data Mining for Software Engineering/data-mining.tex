\documentclass{article}
\usepackage[utf8]{inputenc}
\usepackage[brazil]{babel}
\usepackage{fancyvrb}
\usepackage[margin=0.8in]{geometry}

\title{Resenha: Data Mining for Software Engineering \\
 \large MAT08 - Evolução de Software (2012.2)}
\author{Joenio Marques da Costa}
\date{29 de novembro de 2012}

\begin{document}

\maketitle

\section{Sua resenha deve conter, no mínimo}

Resumo do artigo:

Ideia cntral: Problema relatado e proposto solucao no artigo: aproximacao entre as comunidades de Engenharia de Software e de Mineração de Dados.

Melhorar a produtividade e qualidade de softwares é uma meta importante dentro de Engenharia de Software.

Mineração de dados da Engenharia de Software surge recentemente como um meio promissor para atingit esta meta.

Ferramentas de controle de versao, debuggers e outras tantas permitem Engenheiros de Software coletar uma série dados sobre Software para serem minerados.

Esses dados da Engenharia de Software consiste de 3Ps: pessoas, processos e produtos.

E podem ser categorizados entre: sequencias, grafos e texto.

Engenheiros de Software estao cada vez mais empregando tecnicas de mineracao de dados em varias tarefas da Engenharia de Software.

A metologia no geral se resume a coletar e investigar dados para minerar e entao determinar uma tarefa da ES para auxiliar.

Uma das etapas desta metodologia refere-se a adotar/adaptar/desenvolver um algorithmo de mineracao, este algoritmo usualmente se encaixa em alguma das categorias: frequent pattern mining, pattern matching, clustering, classification.

A ultima etapa transforma os resultados do algoritmo de mineracao em um formato adequado para auxiliar uma tarefa da Engenharia de Software, como localizar petenciais bugs por exemplo.

A mineração de dados em Engenharia de Software apresenta aguns desafios: Requisitos unicos da ES, Dados e padroes complexos, Dados em larga-escala, Mineração just-in-time.

Muitas ferramentas de mineracao de dados nao atendem perfeitamente os requisitos do ES e precisam ser adaptadas, uma aproximacao entre as comunidades pode ser um caminho para solucionar este problema, este é um dos objetivos da pesquisa do autor do artigo.

Os dados brutos disponiveis sao enormes, e ainda carece de algoritmos de mineracao para trabalhar com volume assim.

Na pesquisa foi desenvolvido novos algoritmos de mineracao de dados para auxiliar nas tarefas da Engenharia de Software. *importante*.

Varios projetos de software livre foram estudados, onde aplicou-se alguns dos algoritmos desenvolvidos durante a pesquisa.

Um grupo de algoritmos foi desenvolvido e testado em aplicacoes reais.

Mineracao em grafos, estatico ou dinamico, depende da execucao do programa, outro nao. Tambem foi desenvolvido algoritmos de mineracao em grafos.

Atraves de grafo eh possivel detectar pontos onde ocorre um bug, atraves da  execucao do software sob condicoes que o façam falhar.

Mineração em texto (me parece ser uma área bem complicada, pois n há padrao de documentacao, nem ferramentas de bugreport, etc, etc...).

Atualmente muito trabalho e esforco esta sendo aplicado para adaptar algoritmos de mineracao de uso-geral em algoritmos que atendam os requisitos unicos da Eng de Softw.

Qual o problema resolvido? 

Falta de algoritmos especificos para tratar tarefas da engenheria de software.

Como foi resolvido? 

Aproximacao das comunidades de mineracao com a de engenharia.

Contribuições do artigo.

Algoritmos de mineração de dados aplicados a necessidades especificas a tarefas da Engenharia de Software.

\section{Sua opinião sobre o artigo}

Pontos positivos e negativos.

Nao vejo pontos negativos, pontos positivos e facil ver uma serie de contribuicoes de algoritmos de mineracao e ferramentas.

Você concorda com o artigo? No todo? Em parte? Por quê?

Concordo que uma aproximacao entre as comunidades de mineração de dados e de engenharia de software preencheria o espaço que tem.

\section{E daí?}

Em que e como este artigo mudou sua perspectiva sobre o assunto?

Vejo que grande parte do trabalho da Engenharia de Software experimental se inicia no trabalho de mineracão de dados e que não é possível ignorar a necessidade de se aprofundar um pouco no assunto para desenvolver soluções da Egenharia de Software.

\section{Questões para discussão em sala}

Questoes...

É fundamental estudar e entender técnicas e algoritmos de Mineração de Software para desenvolver Engenharia de Software/Evolução de Software?

Mineracao de texto me pareceu a com mais desafios, isto é verdade?

\section{Referência do artigo}

\begin{itemize}
\item http://en.wikipedia.org/wiki/Dr.\_Watson\_(debugger)
\item http://pt.wikipedia.org/wiki/Mineração\_de\_dados
\item http://pt.wikipedia.org/wiki/Máquina\_de\_estados\_finitos
\end{itemize}

\end{document}
