\documentclass[12pt]{article}
\usepackage[utf8]{inputenc}
\usepackage[brazil]{babel}
\usepackage{fancyvrb}
\usepackage[margin=0.7in]{geometry}
\bibliographystyle{ieeetr}

\title{Resenha: Polymetric Views - A Lightweight Visual Approach to Reverse Engineering\cite{PolymetricViews} \\
 \large MAT08 - Evolução de Software (2012.2)}
\author{Joenio Marques da Costa}
\date{06 de dezembro de 2012}

\begin{document}

\maketitle

\section{Resumo}

A engenharia reversa em grandes sistemas de software tem se tornado um assunto
de grande relevancia devido a importancia no auxílio a tarefas de manutenção,
re-engenharia e evolução de software. Manter grandes sistemas é uma tarefa
inerentemente difícil, estes sistemas são complexos e sofrem de problemas
típicos, como: saída de desenvolvedores, uso de métodos ou linguagens
ultrapassadas, documentação desatualizada, incompleta ou inexistente. Manter e
evoluir estes sistemas tem sido por vezes proibitivamente caros, estudos
mostram que o custo de manutenção pode chegar a 75\% do valor total do projeto,
neste sentido fica claro que soluções de apoio a manutenção e evolução de
softwares são extremamente úteis e desejáveis, desta forma surge uma
metodologia de engenharia reversa baseada em visualizações de software
combinadas com métricas, a esta combinação dá-se o nome de {\it polymetric
views}. Esta metologia traz como resultado final uma visão geral do sistema e
joga luz em quais partes estão boas ou ruins, deixando a cargo do projetista
decidir o que fazer com elas. Estes resultados são obtidos através da um
processo baseado na ferramenta CodeCrawler desenvolvida durante as pesquisas,
esta ferramenta de visualização de software foi desenvolvida com objetivo de
ser extensível, interativa e escalável, além de permitir a verificação de
trechos de código-fonte relacionados aos resultados obtidos.

\section{Questoes}

\begin{itemize}
  \item Esta de visualização chamada {\it polymetric views} foi adotada em outros estudos?
\end{itemize}

\bibliography{bibliografia}
\end{document}
