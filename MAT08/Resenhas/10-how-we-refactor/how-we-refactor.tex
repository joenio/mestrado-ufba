\documentclass[12pt]{article}
\usepackage[utf8]{inputenc}
\usepackage[brazil]{babel}
\usepackage{fancyvrb}
\usepackage[margin=0.7in]{geometry}
\bibliographystyle{ieeetr}

\title{How We Refactor, and How We Know It\cite{HowWeRefactor} \\
 \large MATE08 - Tópicos em Engenharia de Software 1 (2012.2)}
\author{Joenio Marques da Costa}
\date{24 de janeiro de 2013}

\begin{document}

\maketitle

\section*{Resumo}

Refatoração é o processo de alteração da estrutura de um programa sem modificar
seu comportamento. Fowler catalogou em seu seu livro 72 tipos diferentes de
refatoração, variando de mudanças localizadas a mudanças globais no programa,
ele afirma que refatoração traz benefícios significativos ao processo de
desenvolvimento, alguns estudos de caso mostram que a prática de refatoração é
comum entre os desenvolvedores e pode por exemplo melhorar métricas de código.
No entando estas afirmações são em sua maioria tomadas a partir de um único
estudo de caso ou um único método de pesquisa e não podem ser generalizadas.
Este estudo replica parte destes estudos em uma variedade de outros contextos e
explora elementos deixados de lado pela maioria dos autores com o objetivo de
confirmar ou invalidar as conclusões tomadas. Foi utilizado um método
experimental com dados de 4 fontes diferentes, onde aplicou-se várias
estratégias de detecção de refatoração, a fim de testar um grupo de 9 hipóteses
sobre refatoração. Os dados utilizados foram os seguintes: (1) dados capurados
pela ferramenta Mylyn Monitor em 2005 por Murphy e seus colegas sobre o uso do
Eclipse por um grupo de 41 desenvolvedores voluntários; (2) dados publicamente
disponíveis do Eclipse Usage Collector a respeito do perfil de mais de 13 mil
desenvolvedores Java usuários do Eclipse; (3) histórico de refatorações feitas
por 4 desenvolvedores mantenedores de ferramentas de refatoração no Eclipse;
(4) histórico de alterações do código-fonte do Eclipse e jUnit vindos do
sistema de controle de versões CVS. 9 hipóteses sobre o comportamento dos
desenvolvedores durante refatorações foram testadas contra estes dados. De
forma geral conclui-se que os desenvolvedores utilizam muito pouco ferramentas
de apoio a refatoração e a maior parte das refatorações são feitas manualmente.
Acho muito interessante pesquisas que validam estudos anteriores, especialmente
estudos que lidam com cultura e comportamento dos desenvolvedores. Este estudo
foi importante para mim pois confirma minha opinião pessoal sobre o tema, ainda
existe pouca adoção de ferramentas de apoio a refatoração entre os
desenvolvedores.

\bibliography{bibliografia}
\end{document}
