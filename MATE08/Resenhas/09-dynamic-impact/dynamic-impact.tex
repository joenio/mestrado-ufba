\documentclass[12pt]{article}
\usepackage[utf8]{inputenc}
\usepackage[brazil]{babel}
\usepackage{fancyvrb}
\usepackage[margin=0.7in]{geometry}
\bibliographystyle{ieeetr}

\title{An Empirical Comparison of Dynamic Impact Analysis Algorithms
 \cite{DynamicImpactAnalysis} \\
 \large MATE08 - Tópicos em Engenharia de Software 1 (2012.2)}
\author{Joenio Marques da Costa}
\date{17 de janeiro de 2013}

\begin{document}
\maketitle

\section*{Resenha}

Softwares passam por constantes mudanças ao longo de sua vida, estas mudanças
podem causar efeitos indesejados ou mesmo desastrosos no software.  Técnicas
conhecidas como análise de impacto de mudanças em software tem sido
apresentadas na literatura com objetico de predizer potenciais impactos de
tais mudanças, este trabalho apresenta um estudo comparativo de 2 técnicas de
análise de impacto, {\it CoverageImpact} e {\it PathImpact}. Estas técnicas
mostram grande similaridade, mas demonstram ligeiras diferenças quando
aplicadas a programas reais, através de mudanças reais e dados de entradas
reais. Desta forma foi realizado um experimento comparatido entre as técnicas,
onde avaliou-se os custos e a precisão de cada uma, este estudo utilizou
versões diferentes de 3 softwares escritos em Java: NanoXML, Siena e Jaba.
Para cada versão foi feita a análise de impacto de mudança utilizando cada uma
das técnicas e em seguida compararou-se: (a) a precisão dos resultados, (b) o
custo da análise em termos de espaço e (c) o custo da análise em termos de
tempo. Ambas as técnicas requerem uma leve instrumentação para coleta e
armazenamento dos dados provenientes da análise dinâmica do software. Com base
nos dados coletados a partir da instrumentação de cada software foi feito uma
comparação com base em um modelo de processo previamente elaborado para
comparação de técnicas de análise de impacto de mudaça em software. Os
resultados mostraram que em muitos casos PathImpact é mais preciso mas requer
de 7 a 30 vezes mais espaço do que CoverageImpact. Já o tempo de execução de
CoverageImpact se mostrou inferior que PathImpact, indicando que o custo de
análise de impacto em termos de tempo é maior com PathImpact do que com
CoverageImpact. No entando é importante destacar que estes resultados não
podem ser generalizados uma vez que o experimento foi realizado utilizando
apenas 3 softwares como entrada, isto indica a necessidade de replicação deste
experimento em outros contextos a fim de validar tais resultados.

Não vejo análise de impacto baseado em dados provenientes de análise dinâmica
como algo prático a ponto de se tornar amplamente aplicável na indústria de
desenvolvimento de software. Me chamou atenção especial a possibilidade de
relacionar técnicas de análise de impacto com estudos sobre Design Structure
Matrices, ao ponto de comparar por exemplo métricas como Change Cost e
algoritmos de análise de impacto.

\bibliography{bibliografia}
\end{document}
