\documentclass[12pt]{article}
\usepackage[utf8]{inputenc}
\usepackage[brazil]{babel}
\usepackage{fancyvrb}
\usepackage[margin=0.7in]{geometry}
\bibliographystyle{ieeetr}

\title{An Empirical Investigation into the Role of API-Level Refactorings
 during Software Evolution\cite{AnEmpiricalInvestigation} \\
 \large MAT08 - Tópicos em Engenharia de Software 1 (2012.2)}
\author{Joenio Marques da Costa}
\date{10 de janeiro de 2013}

\begin{document}

\maketitle

\section*{Resenha}

Refatoração é o processo de alteração do design de um sistema sem modificar seu
comportamento a fim de melhorar sua manutenção, entendimento e evolução.
Refatorações reduzem o débito técnico e evitam o aumento da complexidade do
sistema ao longo do tempo. A sabedoria popular diz que engenheiros de software
geralmente evitam refatorações em favor de resolução de bugs e desenvolvimento
de novas funcionalidades quando se tem poucos recursos. Alguns estudos
questionam os reais benefícios da refatoração visto que um alto índice de
refatoração vem seguida de um crescente indice de bug reports. Este trabalho
propôe uma investigação sistemática do papel da refatoração no nível da API de
um software durante sua evolução, examinando a relação entre correção de bugs e
ciclo de releases. Um estudo de caso foi realizado entre os projetos Eclipse
JDT, jEdit e Columba, este estudo mostrou algumas conclusões interessantes,
como: após uma refatoração o índice de resolução de bugs cresce, o tempo tomado
para resolver um bug reduziu entre 35\% e 63\% após uma refatoração,
refatorações facilitam a resolução de bugs mas também incluem novos, notou-se
também que modernas IDEs carecem de funcionalidades de apoio aos
desenvolvedores. Outra conclusão importante demonstra que é necessária uma
investigação detalhada sobre os reais benefícios da refatoração e seus impactos
econômicos em relação aos investimentos. Este estudo não é aplicável a qualquer
tipo de refatoração, uma vez que estudou-se apenas refatorações no nível de
API, no entanto o estudo joga luz na relação entre refatoração de API e
correção de bugs durante a evolução do software e conclui que o número de
correções de bugs cresce após refatorações enquanto o tempo de resolução de
bugs decresce. Eu concordo com o estudo feito e não vejo nenhum ponto negativo.

\bibliography{bibliografia}
\end{document}
