\documentclass[11pt]{article}
\usepackage[utf8]{inputenc}
\usepackage[brazil]{babel}
\usepackage{fancyvrb}
%\usepackage[alf]{abntcite}
%\bibliographystyle{abnt-alf}
\bibliographystyle{ieeetr}
%\usepackage[top=0.1in,left=0.5in,right=0.5in,bottom=0.1in]{geometry}

\title{Ferramentas desenvolvidas durante pesquisas em engenharia de software no Brasil: \\
        Uma análise dos últimos 10 anos de SBES (?)
}
\author{Joenio Marques da Costa \\
  {\small Universidade Federal da Bahia (UFBA)} \\
  {\small joenio@colivre.coop.br}
}
\date{\today}

\begin{document}

\maketitle

\begin{abstract}
\dots

Artigo revisao de ferramentas desenvolvidas dentro do contexto ou
durante pesquisas da área de engenharia de software nos últimos 15
anos, quais ferramentas continuam funcionando, quais continuam
recebendo atenção e evoluindo.

Referência, artigo de Paulo e Kon do SBES 2011, fizeram um estudo
revisando os últimos X anos do congresso, trilha ferramentas, avaliou
quais ferramentas foram disponibilizadas e distribuídas como software
livre.

\end{abstract}

\section{Introdução}

Cavalcanti et al\cite{Cavalcanti11} desenvolveram revisão sistemática das
últimas 24 edições do SBES, utilizaram o método proposto por Kitchenham e
Charters\cite{Kitchenham2007}...  O meu estudo seguirá o mesmo formato
desenvolvido no trabalho de Calvacanti\cite{Cavalcanti11}, mas com a mudança
de foco de pesquisas da área de engenharia de software no geral, para
pesquisas em engenharia de software que propõe e/ou desenvolvem ferramentas de
software. Objetivo de identificar informações quantitativas e qualitativas
sobre a produção de ferramentas durantes pesquisas em engenharia de software,
este objetivo é semelhante ao objetivo do trabalho de \cite{Cavalcanti11} mas
com foco em ferramentas. Este trabalho de cavalcanti mostra também que há uma
grande concentração de pesquisas apresentando ferramentas, métodos e processos
submetidos ao SBES entre as edições avaliadas, 1987 à 2010.

Em \cite{Kon11} é citado que foram analisados 6773 projetos usando ferramentas
automatizadas para avaliar a atratividade de um software a partir de métricas
extraídas do código-fonte, este estudo:

A study of the relationships between source code metrics and attractiveness in free software projects

pode talvez ser replicado em projetos desenvolvidos pela comunidade acadêmica,
isso vai avaliar se os projetos sao atrativos ou não.

\cite{Kon11} cita também que ocorre o fato de algumas ferramentas serem
disponibilizadas publicamente, mas o link divulgado não funciona mais, ou está
fora do ar, ou não cita a ferramenta.

Kon11 também cita em sua conclusão que percebeu em seu estudo uma tendência no
aumento de pesquisadores Brasileiros em liberar suas ferramentas como FLOSS,
tendo como benefício para estas pesquisas facilidade de reprodução por outros
pesquisadores, bem como oportunidade de receberem contribuições externas
melhorando suas ferramentas.

A comunidade de engenharia de software encontra dificuldades em utilizar
ferramentas desenvolvidas pela própria academia em suas pesquisas, seja por
dificuldade em obter o software, problemas com licenciamento, setup e
configurações complexas, ou outro.

Como consequência deste problema, podemos citar:

\begin{itemize}
\item Duplicação de esforço, pesquisador não adota ferramentas existentes e
        desenvolve sua própria solução
\item Dificuldade em replicar experimentos, necessário para aumentar a
        validade externa dos estudos. Kon \cite{Kon11} sugere que distribuir
        ferramentas como software livre pode resolver este cenário.
% TODO devo pesquisar sobre estudos que apontam a importancia de
        % reprodubilidade de pesquisas científicas de modo geral para dar
        % lastro a este argumento aqui
\end{itemize}

Para ajudar a entender tal problema será feito um levantamento de
"ferramentas" (*) de software desenvolvidas durante pesquisas acadêmicas em
engenharia de software nos últimos 15 anos. Tais ferramentas serão então
caracterizadas a partir dos seguintes critérios:

\begin{itemize}
\item É utilizada pela comunidade de pesquisa em engenharia de software?
\item É utilizada fora do seu grupo de pesquisa?
\item Há quanto tempo a ferramenta existe?
\item Há quanto tempo continua em uso pela comunidade de pesquisa?
\item Qual problema a ferramenta resolve?
\item É utilizada fora da comunidade acadêmica?
\item É adotada pela indústria?
\item É derivada ou é extensão de alguma outra ferramenta?
\item Tem cobertura de testes?
\item Tem documentação?
\item etc
\item etc
\end{itemize}

% Uma vez encontrada uma ferramenta existe dificuldade em sua adoção/uso?

\begin{itemize}
\item {\it Definir melhor “ferramenta”, o termo está vago.}
\end{itemize}

Isto irá medir a disponibilidade das ferramentas desenvolvidas pela comunidade
acadêmica ao longo do tempo e possivelmente dará indícios de ...

% \section{Coleta e análse dos dados}
% %* o plano de coleta e análise dos dados

\section{Metodologia}

\begin{itemize}
\item Levantar revisões sistemáticas de literatura na área de engenharia de
  software, pode haver algum estudo anterior com a mesma abordagem e que
  responda aos problemas citados aqui
\item Levantar revisões sistemáticas de literatura com foco em “ferramentas"
  (*) para engenharia de software, pode existir estudos similares ou
  complementares ao problema levantado aqui
\item Iniciar revisão sistemática de literatura para responder ao problema
  definido
\end{itemize}

\section{Conclusão}

Kon \cite{Kon11} conclui que construir ferramentas como software livre gera
transferencia de tecnologia em várias vias, uma delas é que proporciona
pesquisadores terem acesso a uma gama enorme de ferramentas existentes já
construídas pela comunidade acadêmica, deixando assim espaço para que o
pesquisador foque seu estudo em sua área de pesquisa, e não se desvie tendo
que desenvolver suas próprias ferramentas, perdendo tempo que poderia ser
usado para se aprofundar mais em suas pesquisas. (esta afirmação de Kon é algo
que eu quero também concluir em minha pesquisa).

% \section{Resumo para publicação}
% 
% Estudos em engenharia de software com foco em experimentação são muito
% importantes\cite{theRoleOfExperimentation}, inúmeras pesquisas tem trazido
% propostas de ambientes e ferramentas para propiciar a replicação dos
% experimentos \cite{towardsAComputerizedInfrastructure} de uma maneira mais
% precisa e simples, no entando ainda não há consenso entre pesquisadores, é
% comum cada pesquisador realizar experimentos ao seu modo, para mitigar tais
% problemas uma proposta recente se baseia no uso ontologias
% \cite{knowledgeSharingIssues} para mapear conceitos entre pacotes de
% laboratórios para experimentos controlados, algumas propostas tem sido
% implementadas neste sentido (DO AMARAL, 2002). No entanto tais propostas e
% modelos parecem bastante ambiciosos pois descrevem cada detalhe de como
% executar e empacotar experimentos, com uma metodologia que deve ser seguida à
% risca por cada pesquisador, isto representa dificuldades
% \cite{knowledgeSharingIssues} na troca de conhecimento entre equipes. Uma
% solução pode ser alcançada através de meta-repositórios
% \cite{usingRepositoryOfRepositories}, ou repositórios contendo meta-dados
% extraídos de repositórios de código-fonte, bugs, emails, etc. Este caminho
% \cite{flossmoleACollaborativeRepository} pode proporcionar um grau de
% facilidade ne replicação de estudos experimentais, aumentando a validade
% interna e externa dos mesmos.

\bibliography{bibliografia}

\end{document}
