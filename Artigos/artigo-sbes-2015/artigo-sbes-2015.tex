\documentclass[conference]{IEEEtran}
\usepackage[utf8]{inputenc}
\usepackage[brazil]{babel}
\usepackage{fancyvrb}
\bibliographystyle{ieeetr}

\title{
  Ferramentas desenvolvidas durante pesquisas em engenharia de software no Brasil:\\
  Uma análise dos últimos 21 anos de SBES
}

\author{
  \IEEEauthorblockN{Joenio Marques da Costa}
  \IEEEauthorblockA{Universidade Federal da Bahia (UFBA)\\
    Email: joenio@colivre.coop.br
  }
}

\begin{document}

\maketitle

%\begin{abstract}
%\dots
%\end{abstract}

% Cavalcanti et al\cite{Cavalcanti11} desenvolveram revisão sistemática das
% últimas 24 edições do SBES, utilizaram o método proposto por Kitchenham e
% Charters\cite{Kitchenham2007}...  O meu estudo seguirá o mesmo formato
% desenvolvido no trabalho de Calvacanti\cite{Cavalcanti11}, mas com a mudança
% de foco de pesquisas da área de engenharia de software no geral, para
% pesquisas em engenharia de software que propõe e/ou desenvolvem ferramentas de
% software. Objetivo de identificar informações quantitativas e qualitativas
% sobre a produção de ferramentas durantes pesquisas em engenharia de software,
% este objetivo é semelhante ao objetivo do trabalho de \cite{Cavalcanti11} mas
% com foco em ferramentas. Este trabalho de cavalcanti mostra também que há uma
% grande concentração de pesquisas apresentando ferramentas, métodos e processos
% submetidos ao SBES entre as edições avaliadas, 1987 à 2010.

% TRABALHO FUTURO
% Em \cite{Kon11} é citado que foram analisados 6773 projetos usando ferramentas
% automatizadas para avaliar a atratividade de um software a partir de métricas
% extraídas do código-fonte, este estudo:
% 
% A study of the relationships between source code metrics and attractiveness in
% free software projects.
% 
% Pode talvez ser replicado em projetos desenvolvidos pela comunidade acadêmica,
% isso vai avaliar se os projetos sao atrativos ou não.

% \cite{Kon11} cita também que ocorre o fato de algumas ferramentas serem
% disponibilizadas publicamente, mas o link divulgado não funciona mais, ou está
% fora do ar, ou não cita a ferramenta.
% 
% Kon11 também cita em sua conclusão que percebeu em seu estudo uma tendência no
% aumento de pesquisadores Brasileiros em liberar suas ferramentas como FLOSS,
% tendo como benefício para estas pesquisas facilidade de reprodução por outros
% pesquisadores, bem como oportunidade de receberem contribuições externas
% melhorando suas ferramentas.

% PROBLEMA QUE TALVEZ EXISTA ENTRE PESQUISAS QUE DESENVOLVEM FERRAMENTAS
% \cite{Leite11} conclui em uma revisão sistemática de literatura de 5 anos
% analizando as publicações do SBES entre os anos de 2006 e 2010, com objetivo
% de dar uma visão geral das publicações recentes do SBES e analisar a
% relevancia para a comunidade e industria, percebeu que há uma incipiente
% colaboração entre universidades nas pesquisas.

\section{Introdução}

Softwares são utilizados em praticamente todas as áreas do conhecimento
humano, sua utilização é crescente em todas elas, no meio acadêmico
pesquisadores utilizam mais e mais softwares para apoiarem suas pesquisas, em
engenharia de software isto se torna particularmente evidente já que
muitas pesquisas resultam também na criação de novos softwares, ou,
novas {\it ferramentas de pesquisa}.

Entender como essas {\it ferramentas de pesquisa} são publicadas e mantidas ao
longo do tempo é crucial para dar os primeiros passos na solução do seguinte
problema: dificuldade enfrentada pelos pesquisadores na utilização de {\it
ferramentas de pesquisa}.

%, a relevância
%desse problema justifica-se pelo impacto causado na replicação de estudos.
%
%constatado no Brasil a partir de um estudo feito em 24 edições do SBES
%mostrando que há uma concentração de
%publicações apresentando ferramentas de software\cite{Cavalcanti11}.
%
%foco deste estudo, levam os pesquisadores a enfrentarem barreiras na
%utilização de {\it ferramentas de pesquisa} em seus estudos, causando dois
%problemas imediatos: duplicação de esforço e dificuldade em replicar
%pesquisas.
%
% \begin{itemize}
% \item Duplicação de esforço
%   %- {\it pesquisador não utiliza ferramenta existente e desenvolve sua própria solução}
% \item Dificuldade em replicar estudos
%   %- {\it pesquisador não consegue reproduzir pesquisas com exatidão}
% %        algo necessário para aumentar a validade externa dos estudos
% %        Kon \cite{Kon11} sugere que distribuir
% %        ferramentas como software livre é requisito para evitar este problema
% % TODO devo pesquisar sobre estudos que apontam a importancia de
%         % reprodubilidade de pesquisas científicas de modo geral para dar
%         % lastro a este argumento aqui
% \end{itemize}

Nesse sentido, será feita uma caracterização das {\it ferramentas de pesquisa}
desenvolvidas nos últimos 21 anos de pesquisa em engenharia de software no
Brasil através de uma revisão sistemática de literatura, esta revisão será
feita a partir de trabalhos submetidos ao Simpósio Brasileiro de Engenharia
de Software (SBES) na trilha ferramentas, tal revisão será guiada pelas
seguintes questões, a serem respondidas para cada ferramenta:

\begin{itemize}
\item Quando foi publicada?
\item Qual pesquisador desenvolveu?
\item Qual grupo de pesquisa desenvolveu?
\item Oferece para a quais sistemas operacionais?
\item Em qual linguagem de programação foi escrita?
\item Qual problema resolve, qual o domínio?
\item É derivada ou é extensão de alguma outra ferramenta?
\item Como se deu a evolução histórica das ferramentas em quantidade, autores
  e organizações no SBES? (questão adaptada do trabalho de
  \cite{Cavalcanti11})
\item Quais os autores mais ativos em publicação de ferramentas no SBES?
  (questão adaptada do trabalho de \cite{Cavalcanti11})
\item Como se deu a evolução de quantidade de ferramentas publicadas no SBES?
  (questão adaptada do trabalho de \cite{Cavalcanti11})
\item Quantos artigos fazem referência à ferramenta? Ou ao artigo que propôs a
  ferramenta?
\end{itemize}

Esta caracterização dará uma visão geral das {\it ferramentas de pesquisa}
em engenharia de software no Brasil, bem como permitirá evidenciar fatores que
causam dificuldades enfrentadas pelos pesquisadores no uso das mesmas.

%\section{Metodologia}
%
%\begin{itemize}
%\item Levantar revisões sistemáticas de literatura de engenharia de software\\
%        {\it pode haver algum estudo anterior com a mesma abordagem e que
%        responda ao mesmo problema}
%\item Levantar revisões sistemáticas de literatura sobre ferramentas de
%        engenharia de software\\
%        {\it pode existir estudos similares ou complementares}
%\item Iniciar revisão sistemática de literatura para responder ao problema
%        definido\\
%        {\it buscar estudos dos últimos 15 anos de pesquisas em engenharia de
%        software}
%\end{itemize}
%
%\section{Conclusão}
%
%\ldots

%Kon \cite{Kon11} conclui que construir ferramentas como software livre gera
%transferencia de tecnologia em várias vias, uma delas é que proporciona
%pesquisadores terem acesso a uma gama enorme de ferramentas existentes já
%construídas pela comunidade acadêmica, deixando assim espaço para que o
%pesquisador foque seu estudo em sua área de pesquisa, e não se desvie tendo
%que desenvolver suas próprias ferramentas, perdendo tempo que poderia ser
%usado para se aprofundar mais em suas pesquisas. (esta afirmação de Kon é algo
%que eu quero também concluir em minha pesquisa).

\bibliography{bibliografia}
\end{document}
