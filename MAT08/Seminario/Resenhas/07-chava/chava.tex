\documentclass[12pt]{article}
\usepackage[utf8]{inputenc}
\usepackage[brazil]{babel}
\usepackage{fancyvrb}
\usepackage[margin=0.7in]{geometry}
\bibliographystyle{ieeetr}

\title{Chava: Reverse Engineering and Tracking of Java Applets
 \cite{Chava} \\
 \large MAT08 - Tópicos em Engenharia de Software 1 (2012.2)}
\author{Joenio Marques da Costa}
\date{17 de janeiro de 2013}

\begin{document}

\maketitle

\section*{Resenha}

A internet começou com servidores provendo apenas arquivos estáticos, depois
com CGI, recentemente Applets Java tem sido crescentemente utilizado para
prover interfaces de usuário ricas e possibilitar processamento no lado
cliente. Enquanto muitas ferramentas de análise de websites estão disponíveis
a maioria despreza código applet. Este trabalho apresenta a ferramenta
Chava, um sistema de engenharia reversa e rastreabilidade para Java.

A ferramenta Java lida tanto com bytecode Java quanto com código-fonte.

Chava também tem a capacidade de comparar 2 versões distintas de um projeto.

Ele combina a análise feita do HTML pelo WebCiao.

O modelo de dados é baseado no modelo entidade-relacionamento de Chen
Cada programa Java é visto como um conjunto de entidades, que referenciam
um ao outro. Uma propriedado do modelo adotado neste estudo é a completude
(completeness), para um modelo ser completo se a compilação de uma entidade
A depende da entidade B, um relacionamento entre elas deve estar no modelo.
A completude não é possível em entidades que implementam reflexão
(metaprogramação) na ferramenta desenvolvida.\cite{DesignSuite}

O modelo proposto armazena as seguintes entidades:
class, interface, package, file, method, field, string.

Toda entidade armazenada no modelo contém os seguintes atributos:
id, name, kind, file, begin line, end line, e chksum.

Além destes existem atributos que são específicos para uma ou outra
entidade.

Entre as entidades é possível ter os seguintes relacionamentos:
subclass, containment, implements, field read, field write, reference.

Usando a ferramenta CIAO é possível visualizar os dados contidos
no modelo de dados gerados pela nossa ferramenta.

Análise de acessibilidade, acessibilidade para a frente e acessibilidade inversa.
Proprociona informação sobre quais entidades serão afetadas por uma certa
mudança.

A ferramenta suporta armazenar informações de várias versões diferentes
de um mesmo projeto, possibilitando verificar entidades incluidas, removidas,
ou entidades que permanecem entre versões/releases.

Chava pode também calcular uma série de métricas a fim de determinar
quão complexo um software é.

...

\bibliography{bibliografia}
\end{document}
