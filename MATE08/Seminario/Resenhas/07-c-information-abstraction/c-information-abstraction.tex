\documentclass[12pt]{article}
\usepackage[utf8]{inputenc}
\usepackage[brazil]{babel}
\usepackage{fancyvrb}
\usepackage[margin=0.7in]{geometry}
\bibliographystyle{ieeetr}

\title{The C Information Abstraction System
 \cite{TheCInformation}\\
 \large MAT08 - Tópicos em Engenharia de Software 1 (2012.2)}
\author{Joenio Marques da Costa}
\date{17 de janeiro de 2013}

\begin{document}

\maketitle

\section*{Resenha}

Analisar a estrutura de grandes programas é uma das mais frustantes e demoradas
partes da manutenção de um programa. Este artigo descreve um programa de abstração
de sistema para apoio durante a manutenção de programas.

A construção de um programa de abstração de sistemas envolve 3 etapas:

1. Formar um modelo conceitual, define os objetos de um software em certa linguagem
e seus relacionamento em certo nível de abstração. O modelo deve ser completo, todos
os objetos e tipos de relacionamentos em um certo nível de abstração estão presentes no
modelo.

2. Extrair visualizações relacionais, um parser é construído para analisar o programa
e extrais informações deacordo com o modelo conceitual.

3. Construir visualizações abstratas, uma visualização abstrata mostra um aspecto
da estrutura do programa e esconde detalhes desnecessários.

Com base nestar premissas foi constrído a ferramente CIA, uma ferramenta de análise
de programas em C. O artigo mostra como a ferramenta pode ser utilizada para estudar
alguns aspectos da estrutura de programas C, como: subsystem, camadas, dead code,
acoplamento.\cite{ExtracaoDeDependencias}

Muitos programas de abstração de sistema sem sido proposto e relatado na literatura,
breve descrição de cada uma, as mais relevantes, e as lições aprendidas com cada uma:

MasterScope (Interlisp), FAST (Fortran), OMEGA (Pascal-like), Cscope (C).

Lições: o processo de extração de informação deve ser separada da fase de apresentação,
um modelo conceitual conciso, um modelo conceitual define quais informações entram
no banco de dados. Separar o banco de dados do código fonte, algumas ferramentas
armazenam texto fonte no banco de dados, isto pode ser um problema para grandes projetos,
o modelo adotado é armazenar no banco de dados apenas uma referencia para o código fonte.
Construção incremental de banco de dados, dividir bancos de dados por arquivos ou
conjunto de arquivos, e prover uma forma de jutar os varios bancos de dados, isto
otimiza a reconstrução do banco de dados quando uma parte do sistema é alterado.


\bibliography{bibliografia}
\end{document}
