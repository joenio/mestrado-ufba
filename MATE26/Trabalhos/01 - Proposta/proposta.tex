\documentclass{article}
\usepackage[utf8]{inputenc}
\usepackage[brazil]{babel}
\usepackage{fancyvrb}
\bibliographystyle{ieeetr}

\title{Título da proposta de trabalho aqui\\
 \large MATE26 - Tópicos Especiais em Engenharia de Software II (2013.1)}
\author{
  Joenio Marques da Costa\\
  \texttt{joenio@colivre.coop.br}
  \and
  Daniela Soares Feitosa\\
  \texttt{daniela@colivre.coop.br}
}
\date{07 de junho de 2013}

\begin{document}

\maketitle

\section*{Resenha}

Criar ferramenta Web para visualização de {\it Design Structure Matrix}
utilizando novas tecnologias HTML5, CSS3 e JavaScript. Esta ferramenta deverá
ter suporte a interação com usuário para visualizar módulos do sistema através
de algoritmos de clusterização de DSM, esta ferramenta deve sugerir mais de
uma opção de cluster e permitir que o usuário visualize cada uma. Por padrão a
DSM deverá ser organizada em módulos que reflitam a organização de arquivos do
sistema, ou seja, fazer cluster por pastas do código-fonte do software sendo
visualizado. Este cluster por pastas é a modularização inerente do software,
proposta pelos seus desenvolvedores. Esta ferramenta irá utilizar como backend
o projeto Analizo, o qual tem suporte a analisar código-fonte C, C++ e Java, e
possui suporte a cálculo de métricas de software, bem como algumas
visualização como a própria DSM. Como frontend será utilizado o framework Web
nodejs que irá se comunicar com o backend para obter os dados que serão
exibidos ao usuário.

Referências:
\cite{AKnowledgeBased,AModelBasedMethod,AnalyzingTheEvolution,AntaresDSM,ApplyingTheDesign,DependencyModel,DesignSuite,EfficientOrganizing,ExploringStructure,PredictingChange,PredictingRequirementChange,ReachabilityMatrices,TheStructureAndValue,UsingTheDesignStructure}.

\bibliography{bibliografia}
\end{document}
