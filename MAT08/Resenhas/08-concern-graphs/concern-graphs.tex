\documentclass[12pt]{article}
\usepackage[utf8]{inputenc}
\usepackage[brazil]{babel}
\usepackage{fancyvrb}
\usepackage[margin=0.7in]{geometry}
\bibliographystyle{ieeetr}

\title{Resenha: Concern Graphs: Finding and Describing Concerns Using
 Structural Program Dependencies\cite{ConcernGraphs} \\
 \large MAT08 - Evolução de Software (2012.2)}
\author{Joenio Marques da Costa}
\date{20 de dezembro de 2012}

\begin{document}

\maketitle

\section*{Resumo}

Alcançar um bom design num sistema de software envolve, entre outras coisas,
numa boa modularização, através da qual facilitará tarefas típicas de
manutenção e evolução, permitindo que desenvolvedores alterem partes do código
de maneira segura sem interferir em outras partes do mesmo sistema.
Infelizmente, frequentemente desenvolvedores se deparam designs de software
inadequados, o que dificulta tarefas de manutenção e evolução, muitas
abordagens estão disponíveis tentar auxiliar o engenheiro de software em
localizar pontos a serem alterados num sistema com design ruim, análise lexica
(grep, etc), ou pesquisa no sistema de controle de versões.  Entretanto todas
essas abordagens levam ao mesmo resultado: o desenvolvedor é apresentado a
linhas de código-fonte, este processo é trabalhoso e dificulta localizar os
conceitos no código-fonte. Neste artigo apresentamos Grafos de Conceitos, uma
representação de conceitos mais efetiva que simples linhas de código-fonte ao
tratar de documentação e análiza de conceitos. O objetivo foi ter uma
representação dos conceitos que requer pouco esforço para criar, manipular e
analizar, e que tenha uma conversão direta de volta ao código-fonte. Para
validar a possibilidade de aplicação de grafo de conceitos criou-se uma
ferramenta chamada FEAT, ela interativamente cria, analiza e visualiza grafos
de programas escrito em Java. Em um cenário comum o desenvolvedor percorre os
arquivos de código-fonte de forma manual em busca dos conceitos, visando
entender onde é preciso fazer uma alteração. Através da ferramenta FEAT o
desenvolvedor apenas monta um modelo com apoio da ferramente que será então
convertido no Grafo de Conceito. Para validar a efetividade da ferramenta na
criacao do Grafo de Conceitos foram feitos vários estudos de caso onde
comprovaram a eficácia dando ao desenvolvedor a visao necessária para fazer a
alteração requerida, todo código analisado para montar o grafo é de interesse
do conceito pesquisado, o numero de falso positivos foi baixo. Não vejo
pontos fracos no artigo e apesar de achar válido não concordo com a
real necessidade de ferramentas de apoio deste tipo aos desenvolvedores,
creio que conhecimento técnico e experiência de desenvolvimento são
os ingredientes comuns para manter e evoluir projetos de software.

\bibliography{bibliografia}
\end{document}
