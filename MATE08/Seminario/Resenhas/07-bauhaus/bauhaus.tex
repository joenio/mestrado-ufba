\documentclass[12pt]{article}
\usepackage[utf8]{inputenc}
\usepackage[brazil]{babel}
\usepackage{fancyvrb}
\usepackage[margin=0.7in]{geometry}
\bibliographystyle{ieeetr}

\title{Bauhaus - a Tool Suite for Program Analysis and Reverse Engineering
 \cite{Bauhaus} \\
 \large MAT08 - Tópicos em Engenharia de Software 1 (2012.2)}
\author{Joenio Marques da Costa}
\date{17 de janeiro de 2013}

\begin{document}

\maketitle

\section*{Resenha}

O projeto Bauhaus foi motivado pelo fato de que os esforços dos desenvolvedores
são em maioria (60\% - 80\%) destinados a manutenção e evolução do que para criar
sistemas. E quase metade do tempo de manutenção é destinado a entender o código
e dados, antes das alterações serem feitas.

O objetivo principal do Beuhaus foi desenvolver meios semi-automatizados de
obter e descrever a arquitetura de um software.

O primeiro desafio da infraestrutura do Bauhaus são o suporte a multiplas linguagens
e a criação de um framework comum.

Component recovery - é o modelo arquitetural, relacionamentos, entidades, etc.

Reflexion analysis - ...

Feature analysis - "concerns"

Protocol analysis - Como as entidades sao utilizadas ao longo da execução...

Static trace extraction - 

Dynamic trace extraction - 

Clone detection - 

Metricas: source code level e arquitecture level.




\bibliography{bibliografia}
\end{document}
