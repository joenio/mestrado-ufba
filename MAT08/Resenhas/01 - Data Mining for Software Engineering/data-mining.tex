\documentclass[12pt]{article}
\usepackage[utf8]{inputenc}
\usepackage[brazil]{babel}
\usepackage{fancyvrb}
\usepackage[margin=0.8in]{geometry}

\title{Resenha: Data Mining for Software Engineering \\
 \large MAT08 - Evolução de Software (2012.2)}
\author{Joenio Marques da Costa}
\date{29 de novembro de 2012}

\begin{document}

\maketitle

\section{Resumo}

A melhoria na produtividade e qualidade de software é uma importante meta da
Engenharia de Software, e a Mineração de Dados de Engenharia de Software tem
sido utilizada cada vez mais para atingir esta meta, a grande quantidade de
dados sobre projetos de software e seus processos de desenvolvimento tornam a
Mineração de Dados de Engenharia de Software num grande aliado na solução de
tarefas comuns como por exemplo localizar petenciais bugs, efetuar manutenção
ou desenvolvedor novos requisitos. No entando, a Mineração de Dados de
Engenharia de Software traz consigo uma série de novos desafios, como a
necessidade de adaptar à mão a maioria das ferramentas de Mineração de Dados
existentes, suprir a demanda por algoritmos de Mineração de Dados capazes de
trabalhar com o grande volume de dados disponíveis, ou mesmo oferecer um meio
de mineração sob demanda onde o tempo de resposta seja imediato. Estes desafios
podem ser melhor enfrentados através de aproximação entre a comunidade de
Mineração de Dados e a comunidade de Engenharia de Software, esta pesquisa
representa uma iniciativa neste sentido que resultou no desenvolvimento de uma
série de novos algoritmos de Mineracao de Dados voltados para tarefas comuns da
Engenharia de Software. A Mineração de Dados de Engenharia de Software em dados
do tipo texto me parece ser um dos grandes desafios desta área. A Mineração de
Dados se mostra como uma área fundamental no estudo da Engenharia de Software,
mais especificamente no estudo sobre evolução de software através da enorme
quandidade de dados disponíveis.

\section{Questoes}

\begin{itemize}
  \item É possível estudar Evolução de Software sem utilizar Mineração de
     Dados?
  \item A mineracao em dados texto me pareceu bem mais desafiadora que a
     mineração em dados de sequência ou grafos.  Isto é verdade?
\end{itemize}

\section{Referências}

\begin{itemize}
  \item http://en.wikipedia.org/wiki/Dr.\_Watson\_(debugger)
  \item http://pt.wikipedia.org/wiki/Mineração\_de\_dados
  \item http://pt.wikipedia.org/wiki/Máquina\_de\_estados\_finitos
\end{itemize}

\end{document}
