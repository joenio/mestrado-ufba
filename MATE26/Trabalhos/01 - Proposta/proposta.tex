\documentclass{article}
\usepackage[utf8]{inputenc}
\usepackage[brazil]{babel}
\usepackage{fancyvrb}
\bibliographystyle{ieeetr}

\title{Visualização de Design Structure Matrix com NodeJS no software
Analizo\\
 \large MATE26 - Tópicos Especiais em Engenharia de Software II (2013.1)}
\author{
  Joenio Marques da Costa\\
  \texttt{joenio@colivre.coop.br}
  \and
  Daniela Soares Feitosa\\
  \texttt{daniela@colivre.coop.br}
}
\date{07 de junho de 2013}

\begin{document}

\maketitle

\section*{Proposta do trabalho}

O objetivo do trabalho será criar uma ferramenta Web para visualização
de {\it Design Structure Matrix (DSM)} utilizando novas tecnologias HTML5,
CSS3  e JavaScript. Esta ferramenta deverá ter suporte a interação com
usuário para visualização dos módulos do sistema através de algoritmos
de clusterização de DSM. Além disso, deve sugerir mais de uma opção de
cluster e permitir que o usuário visualize cada uma das opções.

Por padrão, a DSM deverá ser organizada em módulos que reflitam a organização de
arquivos do sistema, ou seja, fazer cluster por pastas do código-fonte do software
visualizado. Este cluster por pastas é a modularização inerente do software,
proposta pelos seus desenvolvedores.

Esta ferramenta irá utilizar como backend o projeto Analizo, uma suíte
de ferramentas de análise e visualização de código fonte que suporta C,
C++ e Java. Esse projeto possui suporte ao cálculo de mais de 20
métricas de software, além de poder analisar e armazenar tanto métricas
quanto informações estruturais do código-fonte de cada revisão
armazenada em um repositório de controle de versão.

O software Analizo já possui suporte a algumas visualizações, como a
própria DSM, mas nesse trabalho pretendemos implantar uma nova
visualização que permita a interação com o usuário, evoluindo assim o
software.

A interface para interação com o usuário será desenvolvida utilizando o
framework Web NodeJS, uma plataforma para desenvolvimento em Javascript
no lado do servidor que permite respostas rápidas às interações. Após o
usuário inserir o endereço de um repositório de controle de versão
público, o frontend irá se comunicar com o backend, que irá analisar o
código fonte para extração de métricas e obter os dados que serão
exibidos ao usuário.

Referências:
\cite{AKnowledgeBased,AModelBasedMethod,AnalyzingTheEvolution,AntaresDSM,ApplyingTheDesign,DependencyModel,DesignSuite,EfficientOrganizing,ExploringStructure,PredictingChange,PredictingRequirementChange,ReachabilityMatrices,TheStructureAndValue,UsingTheDesignStructure}.

\bibliography{bibliografia}
\end{document}
