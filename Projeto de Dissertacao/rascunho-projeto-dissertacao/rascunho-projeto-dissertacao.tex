\documentclass[12pt]{article}
\usepackage[utf8]{inputenc}
\usepackage[brazil]{babel}
\usepackage{fancyvrb}
\usepackage[alf]{abntcite}
\bibliographystyle{abnt-alf}

\title{Projeto de dissertação (rascunho)}
\author{Joenio Marques da Costa}
\date{Fevereiro de 2015}

\begin{document}

\maketitle

\section{Dados do projeto}

\subsection{Título do projeto}

Ferramentas e serviços de apoio à experimentação em Engenharia de Software: Estado da arte

\subsection{Área do conhecimento}

Exatas, da terra e engenharias

\subsection{Sub-área do conhecimento}

Ciência da computação

\subsection{Palavras chaves}

Experimentação; ferramentas; meta-dados; repositórios.

\section{Objetivo geral}

Avaliação do atual estado da arte de ferramentas, serviços, modelos e
frameworks para apoio a experimentação em Engenharia de Software. Qual nível
de maturidade destas iniciativas, se existem pesquisas atuais usando-as, qual
o nível de adoção entre os pesquisadores, fatores que levam os pesquisadores a
adotarem esta ou aquela ferramenta. Caso nenhuma das iniciativas se apresente
como promissora, apontar qual rumo deve-se tomar para oferecer infra-estrutura
de qualidade para os pesquisadores visando principalmente facilidade de
replicação de experimentos.

\section{Objetivos específicos}

\begin{itemize}
\item Enumerar metodologias para experimentos em engenharia de software
\item Levantar ferramentas e serviços utilizados nos estudos experimentais
\item Avaliar criticamente cada ferramenta e serviço em relação maturidade e adoção
\item Avaliar as vantagens em utilizar meta-dados em detrimentos das ferramentas e serviços
\item Propor repositório de meta-dados caso a avaliação anterior seja positiva
\end{itemize}

\section{Justificativa}

Diversos estudos experimentais em Engenharia de Software utilizam ferramentas
computacionais como apoio, estas ferramentas usualmente extraem e interpretam
dados de repositórios de código- fonte, listas de email, issues tracker,
dentre outras. A replicação de tais estudos nem sempre é possível pela
dificuldade em reproduzir os passos realizados pelos autores, muitas das
ferramentas utilizadas deixam se ser disponibilizadas após algum tempo, outras
vezes os dados do estudo apesar de terem sido disponibilizados durante o
estudo após algum tempo não estão mais disponíveis, dentre outros problemas.
Por conta disso, e partindo do pressuposto que ciência necessita de
experimentação, entende-se que é necessário haver facilidade na replicação de
estudos experimentais.

\section{Revisão da literatura}

\begin{itemize}
\item A Computerized Infrastructure for Supporting Experimentation in Software Engineering
\item An Environment to Support Large Scale Experimentation in Software Engineering
\item Empacotamento de Experimentos Controlados com Abordagem Evolutiva Baseada em Ontologia
\item eSEE – Ambiente de apoio a experimentação em larga escala em Engenharia de Software
\item Experimental Software Engineering: A Report on the State of the Art
\item FLOSSmole: A collaborative repository for FLOSS research data and analyses
\item Infra-estrutura Conceitual para Ambientes de Experimentação em Engenharia de Software
\item Infrastructure for SE Experiments Definition and Planning
\item Introdução à Engenharia de Software Experimental
\item Kalibro: interpretação de métricas de código-fonte
\item Knowledge-Sharing Issues in Experimental Software Engineering
\item Monitoramento de métricas de código-fonte em projetos de software livre
\item The Knowledge Creating Company
\item The Role of Experimentation in Software Engineering: Past, Present, Future
\item Towards a Computerized Infrastructure for Managing Experimental Software Engineering Knowledge
\item Using Repository of Repositories (RoRs) to Study the Growth of FOSS Projects: A Meta-Analysis Research Approach
\end{itemize}

\section{Metodologia}

\begin{itemize}
\item Revisar literatura em busca de propostas de ferramentas e serviços utilizados como apoio em ESE
\item Para cada ferramenta/serviço/repositório avaliar seu estado e seu nível de adoção, maturidade, etc
\item Propor um repositório de meta-dados de código-fonte
\item Avaliar se este repositório atende às necessidades de pesquisa em detrimento do uso das ferramentas usualmente adotadas
\item Definir um rumo para apontar a melhor forma de apoio às diferentes pesquisas em ESE
\end{itemize}

\section{Resultados esperados, motivação e impactos previstos}

Espera-se identificar o atual estado de ferramentas computacionais de apoio a
experimentação, dando foco em experimentação com projetos de software livre,
deixando claro se tais ferramentas atendem as demandas dos pesquisadores ou se
novas ferramentas ou serviços são necessários. Não há interesse em registro de
patentes, qualquer resultado será distribuído abertamente com a comunidade
acadêmica como software livre, acredita-se que isto fortalece a ciência e
contribui com a sociedade de modo geral, partindo do princípio que
conhecimento é poder.

\section{Resumo para publicação}

Estudos em engenharia de software com foco em experimentação são muito
importantes\cite{theRoleOfExperimentation}, inúmeras pesquisas tem trazido
propostas de ambientes e ferramentas para propiciar a replicação dos
experimentos \cite{towardsAComputerizedInfrastructure} de uma maneira mais
precisa e simples, no entando ainda não há consenso entre pesquisadores, é
comum cada pesquisador realizar experimentos ao seu modo, para mitigar tais
problemas uma proposta recente se baseia no uso ontologias
\cite{knowledgeSharingIssues} para mapear conceitos entre pacotes de
laboratórios para experimentos controlados, algumas propostas tem sido
implementadas neste sentido (DO AMARAL, 2002). No entanto tais propostas e
modelos parecem bastante ambiciosos pois descrevem cada detalhe de como
executar e empacotar experimentos, com uma metodologia que deve ser seguida à
risca por cada pesquisador, isto representa dificuldades
\cite{knowledgeSharingIssues} na troca de conhecimento entre equipes. Uma
solução pode ser alcançada através de meta-repositórios
\cite{usingRepositoryOfRepositories}, ou repositórios contendo meta-dados
extraídos de repositórios de código-fonte, bugs, emails, etc. Este caminho
\cite{flossmoleACollaborativeRepository} pode proporcionar um grau de
facilidade ne replicação de estudos experimentais, aumentando a validade
interna e externa dos mesmos.

\bibliography{bibliografia}

\end{document}
