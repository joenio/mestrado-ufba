\documentclass[12pt]{article}
\usepackage[utf8]{inputenc}
\usepackage[brazil]{babel}
\usepackage{fancyvrb}
\usepackage[margin=0.7in]{geometry}
\bibliographystyle{ieeetr}

\title{Resenha: What Makes a Good Bug Report?\cite{GoodBugReport} \\
 \large MAT08 - Evolução de Software (2012.2)}
\author{Joenio Marques da Costa}
\date{13 de dezembro de 2012}

\begin{document}

\maketitle

\section*{Resumo}

{\it Bug reports} são vitais para o processo de desenvolvimento de qualquer
projeto de software, com ele é possível informar aos desenvolvedores sobre
problemas encontrados durante o seu uso, dada a importancia dos {\it bug
reports} foi realizada uma pesquisa de opinião entre desenvolvedores e usuários
de 3 projetos de software livre diferentes com objetivo de avaliar a qualidade
dos {\it bug reports} sob a perspectiva dos desenvolvedores, participaram da
pesquisa 872 desenvolvedores e 1354 usuários. As respostas obtidas demonstram
que há um forte desencontro entre o que os desenvolvedores precisam e o que é
reportado pelos usuários, a fim de reduzir este desencontro foi desenvolvida
uma ferramenta chamada {\it CUEZILLA} com objetivo de apoiar os usuários ao
reportar {\it bugs} incentivando e sugerindo que se forneçam informações mais
completas, ele avalia a qualidade do {\it bug report} e sugere por exemplo que
o usuário adicione um screenshot sobre o problema relatado. As perguntas
oferecidas aos desenvolvedores e usuários foram basicamente as mesmas, e os
resultados mostram que a falta de informação nos {\it bug reports} se dá
principalmente pela dificuldade em coletar os dados necessárias para tal, isto
sugere que ferramentas de apoio neste sentido são extremamente necessárias e o
{\it CUEZILLA} se mostra bastante útil nisso. Apesar das conclusões obtidas é
difícil em estudos empíricos como este esboçar conclusões mais gerais, sabe-se
que é possível por exemplo replicar estas conclusões em outros projetos de
software livre, entretando não se sabe é possível fazer o mesmo em projetos de
software proprietário. Eu achei o artigo bastante interessante e não consigo
discordar em nada do que foi posto, apesar disso o estudo não me trouxe nenhuma
informação nova.

\section*{Questoes}

\begin{itemize}
  \item Análise de informações envolvendo {\it bug reports} parecem sempre
    depender de estudo empíricos, isto é verdade?
\end{itemize}

\bibliography{bibliografia}
\end{document}
