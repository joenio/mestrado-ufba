\documentclass[12pt]{article}
\usepackage[utf8]{inputenc}
\usepackage[brazil]{babel}
\usepackage{fancyvrb}
\usepackage[margin=0.7in]{geometry}
\bibliographystyle{ieeetr}

\title{Evaluating Architectural Extractors
 \cite{EvaluatingArchitectural} \\
 \large MAT08 - Tópicos em Engenharia de Software 1 (2012.2)}
\author{Joenio Marques da Costa}
\date{17 de janeiro de 2013}

\begin{document}

\maketitle

\section*{Resenha}

Um dos objetivos da engenharia reversa de software é extrair design arquitetural
do código fonte.

Ferramentas analisadas:

Rigi - extrai, organiza, abstrai e visualiza grandes volumes de informação.

Dali - Interpreta dados de recuperação arquitetural.

PBS - 

CIA - Banco de dados relacional.

SNiFF+ - Não é originalmente uma ferramente de engenharia reversa mas o seu parser
é comumento utilizado.

O processo de engenharia reversa é dividida em 2 fases: a identificação dos artefatos
de código em baixo nível (extração) e a análise destes artefatos para compor componentes
de algo nível.

Este artigo divide a fase de análise em 2 partes: classificação e visualização. E faz
um estudo das ferramentas para esta função.

Na fase de extração muitas ferramentas apresentaram dificuldades. Quando processando
código C, complexidade da linguagem causam na maioria dos extratores omissão de
informação. Estas dificuldades foram vistas com ponteiros de função, variáveis globais,
e chamada de bibliotecas entre outras.

Estudos mostraram que 2 extratores raramente trazem resultados identicos.

Falso negativo é um relacionamento ou componente que deveriam ser reportados mas não
foi. Falso positivo é um relacionamento ou componente relatado que não existe.

Para testar a capacidade e validade das informações dos extratores estudados foi
desenvolvido um programa exemplo chamado degen foi desenvolvido, contendo
todas as dificuldades de implementacao não bem tratadas por cada extrator estudado.

Uma lista dos problemas encontrados pelos extratores foi documentada e serve
como uma boa base para implementação de um novo extrator, pelo menos em linguagem C.

Entre os extratores analisados o SNiFF+ foi o que mais apresentou bons resultados,
infelizmente ele é um projeto comercial.

\bibliography{bibliografia}
\end{document}
